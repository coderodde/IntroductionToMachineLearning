\documentclass[10pt]{article}

\usepackage[english]{babel}
\usepackage[utf8x]{inputenc}
\usepackage{amsmath}
\usepackage{amssymb}
\usepackage{amsfonts}
\usepackage{graphicx}
\usepackage[ruled,linesnumbered,noend]{algorithm2e}
\usepackage{empheq}
\usepackage{float}
\usepackage{enumitem}
\usepackage{tikz}
\usepackage[colorlinks=true,urlcolor=blue]{hyperref}

\title{Introduction to Machine Learning, Fall 2015 - Exercise session V}
\author{Rodion ``rodde'' Efremov, student ID 013593012}

\begin{document}
 \maketitle

\section*{Problem 2 (3 points)}
\color{blue} \textit{Exclusive or} is a classical example used to demonstrate the limitations of linear classifiers. In the most basic setting, we have two binary inputs and a binary output. Using +1 and -1 as the binary values, the exclusive or function XOR is given by the following table:
\begin{center}
\begin{tabular}{|cc|c|}
$x_1$ & $x_2$ & XOR($x_1, x_2$) \\
\hline
-1 & -1 & -1 \\
-1 & +1 & +1 \\
+1 & -1 & +1 \\
+1 & +1 & -1 \\
\hline
\end{tabular}
\end{center}
\begin{itemize}
\item[(a)] Give a detailed proof for the fact that the exclusive or function cannot be represented as a linear classifier. That is, prove that there is no coefficient vector $(w_0, w_1, w_2)$ such that
\[
\text{XOR}(x_1, x_2) = \text{sign}(w_0 + w_1 x_1 + w_2 + x_2)
\]
whould hold for all possible inputs $(x_1, x_2)$.
\item[(b)] Show that if we include an extra input variable $x_1x_2$, we can represent XOR as a linear classifier in terms of the modified inputs. That is, find a coefficient vector $(w_0, w_1, w_2, w_3)$ such that
\[
\text{XOR}(x_1, x_2) = sign(w_0 + w_1 x_1 + w_2 x_2 + w_3x_1x_2)
\]
holds for all $(x_1, x_2)$.
\end{itemize}
\color{black}

\subsection*{Solution to (a)}
Define three points in $\mathbb{R}^3$:
\begin{description}
\item[$A$] $(-1, -1, w_0 - w_1 - w_2)$,
\item[$B$] $(-1, +1, w_0 - w_1 + w_2)$,
\item[$C$] $(+1, -1, w_0 + w_1 - w_2)$.
\end{description}
We must have
\begin{align*}
& w_0 - w_1 - w_2 < 0, \\
& w_0 - w_1 + w_2 > 0, \\
& w_0 + w_1 - w_2 > 0.
\end{align*}
Next, let us find a plane in $\mathbb{R}^3$ that passes through points $A, B$ and $C$: $z = ax + by + c$. Now,
\[
\begin{cases}
-a - b + c &= w_0 - w_1 - w_2 \\
-a + b + c &= w_0 - w_1 + w_2 \\
a - b + c  &= w_0 + w_1 - w_2,
\end{cases}
\]
which simplifies to
\[
\begin{cases}
-a - b + c &= w_0 - w_1 - w_2 \\
2c &= 2w_0,
\end{cases}
\]
so we have that $c = w_0$. Next,
\[
\begin{cases}
-a - b &= -w_1 - w_2 \\
-a + b & = -w_1 + w_2.
\end{cases}
\]
Which leads to $-2a = -2w_1$ and $a = w_1$, and finally $b = w_2$. Now our plane is defined as
\[
z = w_1x + w_2y + w_3.
\]
The point $D$ has coordinates $(+1, +1, w_0 + w_1 + w_2)$. Since
\begin{align*}
w_0 - w_1 + w_2 &> 0 \\
w_0 + w_1 - w_2 &> 0, \\
\end{align*}
summing the above inequalities we obtain $w_0 > 0$. Also, since $w_0 - w_1 - w_2 < 0$, we have that $w_0 < w_1 + w_2$. So $w_0 + w_1 + w_2 > 0$ and $D$ is ``misclassified'' (we should have had $w_0 + w_1 + w_2 < 0$.). 

If we chose instead of the point $D$, we would get the same contradiction. The problem is that all four points do not lie on a same single plane.
\end{document}