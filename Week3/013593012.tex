\documentclass[10pt]{article}

\usepackage[english]{babel}
\usepackage[utf8x]{inputenc}
\usepackage{amsmath}
\usepackage{amssymb}
\usepackage{amsfonts}
\usepackage{graphicx}
\usepackage[ruled,linesnumbered,noend]{algorithm2e}
\usepackage{empheq}
\usepackage{float}
\usepackage{enumitem}
\usepackage{tikz}
\usepackage[colorlinks=true,urlcolor=blue]{hyperref}

\title{Introduction to Machine Learning, Fall 2014 - Exercise session III}
\author{Rodion ``rodde'' Efremov, student ID 013593012}

\begin{document}
 \maketitle

\section*{Problem 1 (3 points)}

\section*{Problem 2 (3 points)}

\section*{Problem 3 (3 points)}

\section*{Problem 4 (9 points)}
\subsection*{(a)}
\color{blue}
Download the MNIST data from the course web page. In addition to the actual data, the package contains some functions for easily loading the data into Python/Matlab/Octave/R and for displaying digits. See the README files for details. Load the first $N=5,000$ images using the provided function.
\color{black}

\subsection*{Solution to (a)} 
Check!

\subsection*{(b)}
\color{blue}
Use the provided functions to plot a random sample of 100 handwritten digits, and show the associated labels. Verify that the labels match the digit images. (This is a sanity check that you have the data [is] in the right format.)
\color{black}

\subsection*{Solution to (b)}
\begin{center}
\includegraphics[scale=0.5]{100TestMnistImages}
\end{center}
The labels are:
\begin{center}
\begin{verbatim}
0 2 3 9 2 0 6 6 8 6 
9 4 7 5 4 0 4 6 3 8 
7 2 4 5 8 1 1 8 7 6 
9 5 4 6 3 1 3 4 9 3 
0 5 4 2 2 7 8 7 8 4 
5 6 3 7 2 8 9 0 5 9 
0 9 8 1 6 9 3 6 5 8 
9 9 6 8 4 7 2 4 4 4 
7 8 0 0 8 7 8 2 0 8 
6 3 0 1 8 7 3 8 2 1
\end{verbatim}
\end{center}
\end{document}